\documentclass[11pt]{article}

\usepackage{sectsty}
\usepackage{graphicx}


\title{ Gestión de Procesos de requerimientos}
\author{ Dilan Garcia, Patrick Loor, Ariana Chicaiza, Maria Mena, Cijel Espinoza, Bruno Intriago }
\date{\today}

\begin{document}
\maketitle	
\section{Introduccion}

Párrafo 1: La gestión de procesos de requerimientos es una disciplina esencial en el ámbito del desarrollo de software y la ingeniería de sistemas.Los requerimientos son la base sobre la cual se construye cualquier sistema o aplicación, ya que representan las necesidades y expectativas de los usuarios, clientes y stakeholders involucrados.
Párrafo 2: La adecuada gestión de estos requerimientos es crucial para garantizar la entrega de soluciones que cumplan con los criterios de calidad, funcionalidad y satisfacción de los usuarios. En este contexto, surge la importancia de implementar procesos eficientes y estructurados que permitan la captura, el análisis, la validación y la gestión continua de los requerimientos a lo largo de todo el ciclo de vida del proyecto.
\section{Objetivo General}

El objetivo general de este trabajo es analizar y proponer un conjunto de estrategias y técnicas para la gestión efectiva de procesos de requerimientos en proyectos de desarrollo de software y sistemas. Se busca establecer un enfoque sistemático que permita identificar, documentar, evaluar y dar seguimiento a los requerimientos a lo largo del ciclo de vida del proyecto, con el fin de asegurar la satisfacción de los usuarios y la entrega exitosa de soluciones tecnológicas.

\section{Objetivos Especificos}
\begin{enumerate}
\item Análisis de Necesidades: Identificar y comprender las necesidades de los usuarios y stakeholders involucrados en el proyecto, a través de técnicas como entrevistas, encuestas y análisis de documentos, para garantizar una comprensión completa de los requerimientos.
\item Documentación Estructurada: Establecer un proceso de documentación que permita capturar los requerimientos de manera clara, completa y coherente, utilizando técnicas como la especificación de casos de uso, diagramas de flujo y prototipos.
\item Validación y Verificación: Desarrollar estrategias para validar y verificar los requerimientos, involucrando a los usuarios y stakeholders para asegurar que los requerimientos reflejen con precisión sus necesidades y expectativas.
\item Gestión de Cambios: Establecer un sistema para gestionar cambios en los requerimientos a lo largo del proyecto, evaluando el impacto de los cambios propuestos y manteniendo la trazabilidad de las modificaciones.
\item Comunicación Efectiva: Establecer canales de comunicación claros y constantes entre el equipo de desarrollo, los usuarios y stakeholders, para asegurar una comprensión mutua y una retroalimentación constante sobre los requerimientos.
\item Seguimiento Continuo: Implementar mecanismos para el seguimiento continuo de los requerimientos a medida que evoluciona el proyecto, asegurando que las soluciones desarrolladas estén alineadas con los requisitos iniciales. 
\end{enumerate}


\end{document}
